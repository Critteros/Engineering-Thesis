% !TEX root = ../main.tex
\documentclass[../main.tex]{subfiles}

\begin{document}
\section{Automated testing}

Testing ensures that the application works as expected and that new features do not break existing functionality.
During the application development process tests were heavily utilized to ensure that the application behaves as expected.
Primary two types of tests were used: unit tests and integration tests.

Unit tests are used to test the smallest units of code, usually functions or classes. They are used to test the functionality of the code in isolation.
Integration tests were used to test how individual application functionalities work together. Integration tests were performed on the application as a whole, without mocking any of the dependencies.
It also includes spinning up a database and testing the database queries.

\subsection{Environment setup}

The application services are run in Docker containers. The communication with the docker engine is done using the \texttt{Testcontainers} library.
This library provides a simple interface to start container instances while also monitoring and cleaning up the containers after the tests are finished.

An additional layer of abstraction was implemented to further simplify the integration tests. The \texttt{E2ETestManager} class
found in the \texttt{e2e-test-manager.ts} file of the \texttt{backend} application is responsible for starting
and stopping the application services. It also provides performs the database setup which includes applying the database migrations and seeding the database with constant data.
Some operations performed by the \texttt{E2ETestManager} class are shown in the \ref{lst:e2e-test-manager} code snippet.

\begin{listing}[H]
  \tsfile{development/code/setup-example.ts}
  \caption{Example of the \texttt{E2ETestManager} class usage}
  \label{lst:e2e-test-manager}
\end{listing}

This reduces the amount of boilerplate code required to write integration test to only two lines of shown below.

\begin{listing}[H]
  \begin{tscode}
    const manager = new E2ETestManager(TestModule);
    manager.installHooks();
  \end{tscode}
  \caption{Example of the \texttt{E2ETestManager} usage in a test}
\end{listing}

\subsection{Performing authentication in tests}

As most of the application endpoints require authentication, the tests clients need to be authenticated before they can perform requests.
The \texttt{E2ETestManager} class provides a method to perform authentication by making a request to the \texttt{/auth/login} endpoint.

\begin{listing}[H]
  \tsfile{development/code/tests-login.ts}
  \caption{Login method implementation}
\end{listing}

Login method utilizes agent from the \texttt{supertest} library to perform the request. The \texttt{supertest} library is used to test HTTP endpoints.
Making a request to the \texttt{/auth/login} endpoint sets a session cookie which is used to authenticate the client in subsequent requests.

\subsection{GraphQL API tests}

\begin{listing}[H]
  \tsfile{development/code/sample-test.ts}
  \caption{Example of a GraphQL test}
  \label{lst:graphql-test}
\end{listing}

Listing \ref{lst:graphql-test} shows an example of how GraphQL tests are written in the application.
First the authentication is made using the \texttt{login} method from the \texttt{E2ETestManager} class.
Then test data is inserted into the database utilizing \texttt{Prisma} ORM.

The last part is performing the GraphQL request using the manager class and asset the response using test matches from the \texttt{jest} library.

\subsection{Unit tests}

Unit tests are in contrast to the integration tests performed without the manager class. They execute a given function with parameters and asset the result.
These tests often take a table of parameters and expected results as an input and perform the test for each of the parameters, example of which is shown in listing \ref{lst:unit-test}.

\begin{listing}[H]
  \tsfile{development/code/unit-test.ts}
  \caption{Example of a unit test}
  \label{lst:unit-test}
\end{listing}

\end{document}