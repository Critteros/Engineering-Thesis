\documentclass[../main.tex]{subfiles}


\begin{document}
\section{Motivation and Background}
\label{sec:motivation-and-background}

\subsection{Scaling operating system installation to large number of hosts}
The simplest way of installing an operating system on a computer is to use a bootable disk.
Bootable disk are removable media, such as USB flash drives or CD-ROMs, that contains software
which loads the operating system installer or standalone operating systems which does not require installation.

This method is simple and reliable, but it has some drawbacks. Primary, it requires physical access to the computer.
This is not a problem when installing an operating system on a personal computer, but it is a problem when installing
an operating system on remote computers especially when there are many of them. Another problem is that it is not
scalable, because it requires a person to manually install the operating system on each computer individually.

A solution is needed for environments which have to manage large number of computers such a data centers, businesses
or universities. The minimal requirements for such a solution are:
\begin{itemize}
  \item It should allow remote operating system installation. Without physical access to the computer
  \item It should be scalable. Adding new computers to the system should not require manual intervention and be manageable even for non-technical users
  \item The cost of maintaining the system should be low and should not require specialized hardware
  \item Virtualized system should be supported in the same manner as physical systems
\end{itemize}

\subsection{System maintenance and recovery}
Remote operating system installation is not the only problem that system administrators face. They are also
tased with maintaining the system and recovering it in a case of a failure. While most of the failures can be performed
remotely using Secure Shell (SSH) ore Remote Desktop Protocol (RDP), there are some failures that require physical access.
This type of failures are often caused by a failure in the booting process which will result in the operating system
not booting entirely or booting into a state which lacks the necessary tools to recover the system remotely.
These failures called for an on site recovery by a technician, which is both costly and time-consuming.
As long as the computer is able to properly communicate with the network there is another solution for this category
of failures. The solution involves booting computer into a specialized minimal operating system which provides
the necessary tools to recover the system remotely. This type of operating systems are called Live CD or Live USB.
They differ from the standard operating systems in that they do not require installation and can operate without
a persistent storage such as a hard drive or a solid state drive. They are loaded into the memory of the computer.
One of the most popular recovery Live CD is SysRescueCD \cite{sysrescuecd}.




\end{document}