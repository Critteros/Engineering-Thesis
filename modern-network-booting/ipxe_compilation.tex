% !TEX root = ../main.tex
\documentclass[../main.tex]{subfiles}
\externaldocument{../../introduction/build/introduction}

\begin{document}

\section{iPXE firmware compilation}

The official iPXE documentation provides a quick overview of the compilation process \cite{ipxe_compilation_guide},
alongside detailed explanations of the various build options \cite{ipxe_build_options}.

For usage in the project, the following build options were enabled by modifying the \texttt{src/config/general.h} file:

\begin{itemize}
  \item \texttt{REBOOT\_CMD}   - Support for the \texttt{reboot} command
  \item \texttt{CONSOLE\_CMD}  - Support for the \texttt{console} command which makes it possible to display background images
  \item \texttt{POWEROFF\_CMD} - Support for the \texttt{poweroff} command
  \item \texttt{PING\_CMD}     - Support for the \texttt{ping} command, mostly used for debugging network connectivity
  \item \texttt{IP\_STAT\_CMD} - Support for the \texttt{ipstat} command, this command is used to display network statistics like number of received packets
\end{itemize}

To properly compile the iPXE firmware, the following build dependencies are required:

\begin{itemize}
  \item \texttt{gcc}      - The GNU Compiler Collection
  \item \texttt{binutils} - A collection of binary tools
  \item \texttt{make}     - Tool used for automating the build process
  \item \texttt{perl}     - A programming language used for various build scripts
  \item \texttt{liblzma}  - Compression library used for compressing the firmware image
\end{itemize}

Method of installing mentioned dependencies varies depending on the operating system and distribution used.
For debian-based distribution the following command can be used:

\begin{listing}[H]
  \bashfile{modern-network-booting/install-build-deps.sh}

  \caption{Installing build dependencies on Debian-based distributions}
\end{listing}

After installing the required dependencies, the firmware can be compiled by running
one of the provided make targets \cite{ipxe_build_targets}, notably \texttt{bin-x86\_64-efi/ipxe.efi} for UEFI
booting on the x86\_64 architecture and \texttt{bin/undionly.kpxe} for legacy BIOS booting.

Additionaly the \texttt{bin/ipxe.iso} target can be used to create a bootable ISO image containing the iPXE firmware.
This image can be used to boot the firmware on a virtual machine or a physical machine by burning it to a CD or a USB drive.
An additional ISO target for flashing the firmware onto a USB drive is also provided (\texttt{bin/ipxe.usb}).

\begin{listing}[H]
  \begin{bashcode}
    make -j $(nproc --all) bin-x86_64-efi/ipxe.efi
  \end{bashcode}

  \caption{Compiling the iPXE firmware for UEFI booting on the x86\_64 architecture}
\end{listing}

As by default the \texttt{make} command only uses a single
thread for parallel compilation, the \texttt{-j} flag is used to specify the number of threads to use.
The \texttt{nproc --all} command is used to determine the number of available CPU cores.
For some cases it might be necessary to reduce the number of cores used for compilation
to avoid putting too much strain on the system.

The firmware image can be found in the \texttt{bin-x86\_64-efi/ipxe.efi} directory.
Besides the firmware image, this directory also contains many temporary files used for the compilation process (e.g. object files)
which can be safely deleted.

\begin{listing}[H]
  \begin{bashcode}
    rm -rf bin-x86_64-efi/*.o
  \end{bashcode}

  \caption{Delete temporary files generated during the compilation process}
\end{listing}



\end{document}