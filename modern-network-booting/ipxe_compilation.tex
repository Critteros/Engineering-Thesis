% !TEX root = ../main.tex
\documentclass[../main.tex]{subfiles}
\externaldocument{../../introduction/build/introduction}

\begin{document}

\section{iPXE firmware compilation}

The official iPXE documentation provides a quick overview of the compilation process \cite{ipxe_compilation_guide},
alongside detailed explanations of the various build options \cite{ipxe_build_options}.

For usage in the project, the following build options were enabled by modifying the \texttt{src/config/general.h} file:

\begin{itemize}
  \item \texttt{REBOOT\_CMD}   - Support for the \texttt{reboot} command
  \item \texttt{CONSOLE\_CMD}  - Support for the \texttt{console} command which makes it possible to display background images
  \item \texttt{POWEROFF\_CMD} - Support for the \texttt{poweroff} command
  \item \texttt{PING\_CMD}     - Support for the \texttt{ping} command, mostly used for debugging network connectivity
  \item \texttt{IP\_STAT\_CMD} - Support for the \texttt{ipstat} command, this command is used to display network statistics like number of received packets
\end{itemize}

To properly compile the iPXE firmware, the following build dependencies are required:

\begin{itemize}
  \item \texttt{gcc}      - The GNU Compiler Collection
  \item \texttt{binutils} - A collection of binary tools
  \item \texttt{make}     - Tool used for automating the build process
  \item \texttt{perl}     - A programming language used for various build scripts
  \item \texttt{liblzma}  - Compression library used for compressing the firmware image
\end{itemize}

Method of installing mentioned dependencies varies depending on the operating system and distribution used.
For debian-based distribution the following command can be used:

\begin{code}
  \bashfile{modern-network-booting/install-build-deps.sh}

  \captionof{listing}{Installing build dependencies on Debian-based distributions}
\end{code}




\end{document}