\documentclass[../main.tex]{subfiles}
\externaldocument{../../introduction/build/introduction}


\begin{document}
\section{iPXE Introduction}
\label{sec:ipxe-introduction}

The PXE standard explained in subsection \ref{subsec:network-booting} have been implemented notably
by the \texttt{gPXE} and \texttt{iPXE} projects. \texttt{gPXE} was the first implementation of the PXE standard
and was later surpassed by \texttt{iPXE} which is a fork of \texttt{gPXE} \cite{ipxe_vs_gpxe}.

\texttt{iPXE} project is still actively maintained and open-sourced under GNU GPL v2 license \cite{ipxe_license}.
It provides additional features that go beyond the PXE specification, most notably the ability to boot from
a web server via \texttt{HTTP} with optional \texttt{HTTPS} support. This feature alongside the added support
for controlling the boot process with a specialized scripting language called \texttt{iPXE} scripts, makes
\texttt{iPXE} it an ideal choice when implementing a network booting solution.

The \texttt{iPXE} firmware can be flushed to a network card's ROM chip, loaded from a bootable media
or executed as a bootloader in existing \texttt{PXE} environment as shown in figure \ref{fig:pxe_protocol}.

\end{document}