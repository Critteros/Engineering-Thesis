% !TEX root = ../main.tex
\documentclass[../main.tex]{subfiles}

\begin{document}
\section{iPXE scripting}

\subsection{Scripting overview}

The iPXE firmware comes with a built-in scripting language that allows for a wide range of customizations.
This language resembles a shell scripting language, but is more limited in its capabilities.
The primary operations that can be performed are:

\begin{itemize}
  \item \textbf{Variable assignment}: Variables can be set and read, and can be used to store configuration values
  \item \textbf{Control flow}: iPXE supports basic control flow statement based on the \texttt{goto} command and labels
  \item \textbf{Configuration}: iPXE scripting API allows for configuration of network interfaces with both static and dynamic IP addresses
  \item \textbf{Certificate management}: iPXE scripting API allows for management of certificates and keys, this allows for secure communication with HTTPS servers and loading only trusted scripts
  \item \textbf{HTTP requests}: Scripting language supports making HTTP requests, although the usage is limited to downloading scripts and files
  \item \textbf{Basic user interaction via console}: iPXE scripting comes with a set of commands that allow basic I/O (Input and Output) operations via the console
\end{itemize}

Comments in iPXE scripting are prefixed with the \texttt{\#} character and are ignored by the interpreter.
It is important to start all iPXE scripts with a comment that contains the \texttt{\#!ipxe} string.
The hash character followed by an exclamation mark is called a shebang. It is used to indicate
how the file should be interpreted when executing this file.

By default, the iPXE interpreter will try to interpret any file
as a binary executable which will lead to crash in the pre boot environment which will lead to a system reboot.

\subsection{Variables}

The iPXE script primarily serves to set up the boot process, and it utilizes variables to store the configuration.
Variables are stored as arrays of bytes \cite{ipxe_settings_types_docs}, when working with variables for most cases it is sufficient to treat them as strings.
All variables are accessible globally and can be read and written to at any time.
The detailed list of variable types is not documented in the official documentation \cite{ipxe}, but can be found in the source code \cite{ipxe_settings_types_docs}.


The scripting environment comes with a set of predefined variables that hold the current configuration of the system.
For example, the \texttt{net0/ip} variable holds the IP address of the first network interface. Setting
this variable will change the IP address of the first network interface.
The variables can be accessed using the \texttt{set} and \texttt{isset} commands.
The \texttt{set} command takes two arguments, the first argument is the name of the variable and the second argument is the value that the variable will be set to.
The \texttt{isset} command takes one argument, the name of the variable, and returns 1 if the variable is set and 0 otherwise.
For accessing the value of a variable, the \texttt{\$\{variable\_name\}} syntax is used.

\begin{listing}[H]
  \textfile{modern-network-booting/code/variable_operations.ipxe}
  \caption{Basic variable operations in iPXE scripts}
\end{listing}

To display the value of a variable, either the \texttt{echo} or \texttt{show} commands can be used.
The \texttt{echo} command will print the value that was passed to it, while the \texttt{show} command
will use the passed value as the name of the variable and print its value alongside variable type.

\begin{listing}[H]
  \textfile{modern-network-booting/code/variable_types.ipxe}
  \caption{Setting and displaying variable types in iPXE scripts, the commands are executed in the iPXE shell}
\end{listing}


Although using string values for variables might suffice, there are situations where it becomes beneficial to incorporate additional validation to ensure that the value aligns with the correct type.
The type of variable can be explicitly provided by suffixing the value with the colon character (\texttt{:}) and the type of the variable.
When using a given variable for a value of another variable, the resulting type of the variable will be the same as the type of the variable that is currently being set, this
is demonstrated in listing \ref{code:variable_types_when_assigning}.

\begin{listing}[H]
  \textfile{modern-network-booting/code/variable_types_when_assigning.ipxe}
  \caption{How types of variables are inferred when assigning values to variables}
  \label{code:variable_types_when_assigning}
\end{listing}

Some commands require the value to be of a specific type, most notable the increment (\texttt{inc}) and decrement (\texttt{dec}) commands, this is demonstrated in listing \ref{code:increment_type_safety}.

\begin{listing}[H]
  \textfile{modern-network-booting/code/increment_type_safety.ipxe}
  \caption{Increment command interactions with values of different types}
  \label{code:increment_type_safety}
\end{listing}

\subsection{Control flow}
All ipxe scripts are executed sequentially, line by line.
Each line is broken down into commands and arguments and executed in the defined order.

To define multiple commands in the same line one of the following commands separators must be used.

\begin{itemize}
  \item \texttt{;}    - Executes the next command regardless of the result of the previous command
  \item \texttt{\&\&} - Executes the next command only if the previous command was successful
  \item \texttt{||}   - Executes the next command only if the previous command failed
\end{itemize}

\begin{listing}[H]
  \textfile{modern-network-booting/code/control_flow_example.ipxe}
  \caption{Basic control flow operations in iPXE scripts}
\end{listing}

Combining the usage of \texttt{\&\&} and \texttt{||} allows for basic
if-else logic to be implemented in the iPXE scripting language.
As these operators rely on the return value of the previous command,
they can be used for basic error handling. Return value of 0 indicates success,
while any other value indicates failure.

To define more complex control flow, the \texttt{goto} command can be used.
The \texttt{goto} command takes one argument, the name of the label to jump to.
Labels are defined using the \texttt{:} character followed by the name of the label.
The \texttt{goto} command can be used to implement loops and conditional statements.

\begin{listing}[H]
  \textfile{modern-network-booting/code/goto_example.ipxe}
  \caption{Infinite loop implemented using goto command}
\end{listing}

Any kind of control flow can be implemented by combining the \texttt{goto} command with variables.

\begin{listing}[H]
  \textfile{modern-network-booting/code/for_loop_example.ipxe}
  \caption{"For loop" construct implemented in iPXE scripts}
  \label{code:for_loop_example}
\end{listing}

Listing \ref{code:for_loop_example} demonstrates how a "for loop" can be implemented in iPXE scripts. As shown in the example this kind of loop can be used to
iterate over network interfaces and perform operations on them.

\end{document}