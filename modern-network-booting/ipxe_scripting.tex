% !TEX root = ../main.tex
\documentclass[../main.tex]{subfiles}

\begin{document}
\section{iPXE scripting}

\subsection{Scripting overview}

The iPXE firmware comes with a built-in scripting language that allows for a wide range of customizations.
This language resembles a shell scripting language, but is more limited in its capabilities.
The primary operations that can be performed are:

\begin{itemize}
  \item \textbf{Variable assignment}: Variables can be set and read, and can be used to store strings
  \item \textbf{Control flow}: iPXE supports basic control flow statement based on the \texttt{goto} command and labels
  \item \textbf{Configuration}: iPXE scripting API allows for configuration of network interfaces with both static and dynamic IP addresses
  \item \textbf{Certificate management}: iPXE scripting API allows for management of certificates and keys, this allows for secure communication with HTTPS servers and loading only trusted scripts
  \item \textbf{HTTP requests}: Scripting language supports making HTTP requests, although the usage is limited to downloading scripts and files
  \item \textbf{Basic user interaction via console}: iPXE scripting comes with a set of commands that allow basic I/O (Input and Output) operations via the console
\end{itemize}

Comments in iPXE scripting are prefixed with the \texttt{\#} character and are ignored by the interpreter.
It is important to start all iPXE scripts with a comment that contains the \texttt{\#!ipxe} string.
The hash character followed by an exclamation mark is called a shebang. It is used to indicate
how the file should be interpreted when executing this file.

By default, the iPXE interpreter will try to interpret any file
as a binary executable which will lead to crash in the pre boot environment which will lead to a system reboot.

\subsection{Variables}

The iPXE script primarily serves to set up the boot process, and it utilizes variables to store the configuration.
Variables are stored as strings and there do not exist any scoping rules like in other programming languages.
All variables are accessible globally and can be read and written to at any time.

The scripting environment comes with a set of predefined variables that hold the current configuration of the system.
For example, the \texttt{net0/ip} variable holds the IP address of the first network interface. Setting
this variable will change the IP address of the first network interface.
The variables can be accessed using the \texttt{set} and \texttt{isset} commands.
The \texttt{set} command takes two arguments, the first argument is the name of the variable and the second argument is the value that the variable will be set to.
The \texttt{isset} command takes one argument, the name of the variable, and returns 1 if the variable is set and 0 otherwise.
For accessing the value of a variable, the \texttt{\$\{variable\_name\}} syntax is used.

\begin{listing}[H]
  \textfile{modern-network-booting/code/variable_operations.ipxe}
  \caption{Basic variable operations in iPXE scripts}
\end{listing}

To display the value of a variable, either the \texttt{echo} or \texttt{show} commands can be used.
The \texttt{echo} command will print the value that was passed to it, while the \texttt{show} command
will use the passed value as the name of the variable and print its value.


\subsection{Control flow}
All ipxe scripts are executed sequentially, line by line.
Each line is broken down into commands and arguments and executed in the defined order.

To define multiple commands in the same line one of the following commands separators must be used.

\begin{itemize}
  \item \texttt{;}    - Executes the next command regardless of the result of the previous command
  \item \texttt{\&\&} - Executes the next command only if the previous command was successful
  \item \texttt{||}   - Executes the next command only if the previous command failed
\end{itemize}

\begin{listing}
  \begin{textcode}
    # Set the variable
    set my_variable "Hello World"
    # Check if the variable is set and print a message if it is
    isset ${my_variable} && echo "Variable is set" || echo "Variable is not set"
      # Print the variable
      echo ${my_variable}
    # Show the value of the variable
    show my_variable
  \end{textcode}
\end{listing}

\end{document}